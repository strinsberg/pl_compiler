\documentclass{article}
\usepackage{fullpage}


\begin{document}

\title{CPSC 4660 PL Compiler}
\author{Steven Deutekom\\
	Ricky Bueckert\\
	University of Lethbridge}
\date{\today}

\maketitle

\section{Contributions}
The entire class contributed equally on this project. At each phase we split
the work up evenly. Until the current work from home pandemic response we
collaborated in the lab whenever we worked. This involved deciding what to
work on next when a portion of the project was finished. It also involved
debugging and design decisions. It is always helpful to have another person
to bounce ideas off of and a fresh pair of eyes when code is not working. We
feel the team aspect was valuable both for finishing the project and for learning
about team work for our future jobs and research.

\section{Files Included}

\subsection{Header Files}
\begin{itemize}
  \item Symbol.h
  \item SymbolTable.h
  \item Token.h
  \item Scanner.h
  \item Administration.h
  \item Parser.h
  \item Grammar.h
  \item Types.h
  \item BlockTable.h
  \item Assembler.h - Provided by professor
\end{itemize}

\subsection{Source Files}
\begin{itemize}
  \item plc.cc
  \item SymbolTable.cc
  \item Token.cc
  \item Scanner.cc
  \item Administration.cc
  \item Parser.cc
  \item BlockTable.cc
  \item Assembler-2.cc - Provided by professor
\end{itemize}

\subsection{Test Files}
\begin{itemize}
  \item search.pl - Search a list of integers for a given integer
  \item fact.pl - Recursivley calculate factorials
  \item recursion.pl - Simple recursion to make a given number 0
  \item arith.pl - Tests the basic arithmetic operators given 2 numbers
  \item bool.pl - Tests the boolean operators
  \item table.pl - Input a table and it is printed out again
  \item maxmin.pl - Finds the min and max number in a given list of 5 integers

  \item errorTest.pl - Tests a number of different errors. Tries to ensure that the parser catches all errors and recovers to a stable state.
  \item typeErrors.pl - Check a few of the possible type errors.

  \item p2.pl, simple\_proc.pl - Provided by professor to ensure proper output of parser
  \item driver.cc, interp.h, interp.cc - Provided by professor for the interpreter
\end{itemize}

\section{Compiling and Running}
\begin{enumerate}
	\item Navigate into the project folder and type the following:

\verb|        $ make|

	\item This will create a binary called compiler that can be run with the following command:

\verb|        $ compiler <testfile-path> -o <output-file>|

	\item If you omit the \verb|-o <output-file>| flag the tokens will be output to a file called \verb|pl.out|
	\item If you add the \verb|-v| flag some debugging info will be printed. This will include each function that is called and matched tokens. This will not affect the output of tokens.
    \item In addition to the compiler an interpreter will be created as well. It can be run as follows:

\verb|        $ interpret <compiled-file-path>|
    \item When running programs \verb|Input:| allows you to enter a single integer.
    \item Output from programs will be displayed with \verb|Output:|
\end{enumerate}

\section{Bugs}
No bugs. However, while testing in the final phase we discovered our grammar does
not allow for negative constants. The unary minus is part of the simpleExpression
rule in the grammar. Constant definitions do not use this rule, but use the
constant rule. The constant rule only deals with numbers, bools, and identifiers.
So without some changes to the grammar for now there are no negative constants.
Though it is possible to use negative numbers everywhere else.

\section{Time Spent}
Hours spent on this project combine all group member contributions. Scanner 30 hrs. Parser 40 hours.
Rule additions and Scope and Type checking 30 hours. Code generation 20 hours.

\end{document}
